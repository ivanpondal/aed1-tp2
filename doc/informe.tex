% Clase y configuracion de tipo de documento
\documentclass[10pt,a4paper,spanish]{article}
% Inclusion de paquetes
\usepackage{a4wide}
\usepackage{amsmath, amscd, amssymb, amsthm, latexsym}
\usepackage[spanish]{babel}
\usepackage[utf8]{inputenc}
\usepackage[width=15.5cm, left=3cm, top=2.5cm, height= 24.5cm]{geometry}
\usepackage{fancyhdr}
\pagestyle{fancyplain}
\usepackage{listings}
\usepackage{enumerate}
\usepackage{xspace}
\usepackage{longtable}
\usepackage{caratula}
\usepackage{xcolor}
% incluye macros espec materia
%practicas
\newcommand{\practica}[2]{%
    \title{Pr\'actica #1 \\ #2}
    \author{Algoritmos y Estructuras de Datos I}
    \date{Segundo Cuatrimestre 2014}

    \maketitle
}

%ejercicios
\newtheorem{exercise}{Ejercicio}
\newenvironment{ejercicio}{\begin{exercise}\rm}{\end{exercise} \vspace{0.2cm}}
\newenvironment{items}{\begin{enumerate}[i)]}{\end{enumerate}}
\newenvironment{subitems}{\begin{enumerate}[a)]}{\end{enumerate}}
\newcommand{\sugerencia}[1]{\noindent \textbf{Sugerencia:} #1}

%tipos basicos
\newcommand{\float}{\mathsf{Float}}
\newcommand{\rea}{\mathbb{R}}
\newcommand{\ent}{\mathbb{Z}}
\newcommand{\cha}{\mathsf{Char}}
\newcommand{\bool}{\mathsf{Bool}}
\newcommand{\Then}{\rightarrow}
\newcommand{\Iff}{\leftrightarrow}

\newcommand{\mcd}{\mathrm{mcd}}
\newcommand{\prm}[1]{\ensuremath{\mathsf{prm}(#1)}}
\newcommand{\sgd}[1]{\ensuremath{\mathsf{sgd}(#1)}}

%listas
\newcommand{\TLista}[1]{[#1]}
\newcommand{\lvacia}{\ensuremath{[\ ]}}
\newcommand{\longitud}[1]{\left| #1 \right|}
\newcommand{\cons}[1]{\mathsf{cons}(#1)}
%\newcommand{\cons}[1]{\mathrm{cons}(#1)}
\newcommand{\indice}[1]{\mathsf{indice}(#1)}
%\newcommand{\indice}[1]{\mathrm{indice}(#1)}
\newcommand{\conc}[1]{\mathsf{conc}(#1)}
%\newcommand{\conc}[1]{\mathrm{conc}(#1)}
\newcommand{\cab}[1]{\mathsf{cab}(#1)}
%\newcommand{\cab}[1]{\mathrm{cab}(#1)}
\newcommand{\cola}[1]{\mathsf{cola}(#1)}
%\newcommand{\cola}[1]{\mathrm{cola}(#1)}
\newcommand{\sub}[1]{\mathsf{sub}(#1)}
%\newcommand{\sub}[1]{\mathrm{sub}(#1)}
\newcommand{\en}[1]{\mathsf{en}(#1)}
%\newcommand{\en}[1]{\mathrm{en}(#1)}
\newcommand{\cuenta}[2]{\mathsf{cuenta}\ensuremath{(#1, #2)}}
%\newcommand{\cuenta}[2]{\ensuremath{\mathrm{cuenta}(#1, #2)}}
\newcommand{\suma}[1]{\mathsf{suma}(#1)}
%\newcommand{\suma}[1]{\mathrm{suma}(#1)}
\newcommand{\twodots}{\mathrm{..}}

% Acumulador
\newcommand{\acum}[1]{\mathsf{acum}(#1)}

% \selector{variable}{dominio}
\newcommand{\selector}[2]{#1~\ensuremath{\leftarrow}~#2}
\newcommand{\selec}{\ensuremath{\leftarrow}}

% -----------------
% Especificacion
% -----------------

% Para problemas con resultado:
% \begin{problema}{nombre}{argumentos}{resultado}
%       \modifica{variables}
%       \requiere[nombre]{condicion}
%       \requiere[nombre]{condicion}
%       ...
%       \asegura[nombre]{condicion}
%       \asegura[nombre]{condicion}
%       ...
% \end{problema}

% Para problemas sin resultado:
% \begin{problema*}{nombre}{argumentos}
%       \modifica{variables}
%       \requiere[nombre]{condicion}
%       \requiere[nombre]{condicion}
%       ...
%       \asegura[nombre]{condicion}
%       \asegura[nombre]{condicion}
%       ...
% \end{problema*}

\newenvironment{problema}[3]{
    \vspace{0.2cm}
    \noindent \textsf{problema #1}\ensuremath{(#2) = #3\{}\\
    \begin{tabular}{p{0.02\textwidth} p{0.85 \textwidth}}
}{
    \end{tabular}

    \noindent \ensuremath{\}}
    \vspace{0.15cm}
}

\newenvironment{problema*}[2]{
    \vspace{0.2cm}
    \noindent \textsf{problema #1}\ensuremath{(#2)\{}\\
    \begin{tabular}{p{0.02\textwidth} p{0.85 \textwidth}}
}{
    \end{tabular}

    \noindent \ensuremath{\}}
    \vspace{0.15cm}
}

\newcommand{\requiere}[2][]{& \textsf{requiere #1: }\ensuremath{#2};\\}
\newcommand{\asegura}[2][]{& \textsf{asegura #1: }\ensuremath{#2};\\}
\newcommand{\modifica}[1]{& \textsf{modifica }\ensuremath{#1};\\}
\newcommand{\pre}[1]{\textsf{pre}\ensuremath{(#1)}}
\newcommand{\aux}[2]{& \textsf{aux }\ensuremath{#1 = #2};\\}
\newcommand{\problemanom}[1]{\textsf{#1}}
\newcommand{\problemail}[3]{\textsf{problema #1}\ensuremath{(#2) = #3}}
\newcommand{\problemailsinres}[2]{\textsf{problema #1}\ensuremath{(#2)}}
\newcommand{\requiereil}[2][]{\textsf{requiere #1: }\ensuremath{#2}}
\newcommand{\asegurail}[2][]{\textsf{asegura #1: }\ensuremath{#2}}
\newcommand{\modificail}[1]{\textsf{modifica }\ensuremath{#1}}
\newcommand{\auxil}[2]{\textsf{aux }\ensuremath{#1 = #2};}
\newcommand{\auxnom}[1]{\textsf{aux }\ensuremath{#1}}

% -----------------
% Tipos compuestos
% -----------------

\newcommand{\Pred}[1]{\mathit{#1}}
\newcommand{\TSet}[1]{\textsf{Conjunto}\ensuremath{\langle #1 \rangle}}
\newcommand{\TSetFinito}[1]{\textsf{Conjunto}\ensuremath{\langle #1 \rangle}}
\newcommand{\TRac}{\tiponom{Racional}}
\newcommand{\TVec}{\tiponom{Vector}}
\newcommand{\True}{\mathrm{True}}
\newcommand{\False}{\mathrm{False}}
\newcommand{\Func}[1]{\mathrm{#1}}
\newcommand{\cardinal}[1]{\left| #1 \right|}
\newcommand{\enum}[2]{\ensuremath{\mathsf{#1} = \langle \mathsf{#2} \rangle}}

\newenvironment{tipo}[1]{%
    \vspace{0.2cm}
    \textsf{tipo #1}\ensuremath{\{}\\
    \begin{tabular}[l]{p{0.02\textwidth} p{0.02\textwidth} p{0.82 \textwidth}}
}{%
    \end{tabular}

    \ensuremath{\}}
    \vspace{0.15cm}
}

\newcommand{\observador}[3]{%
    & \multicolumn{2}{p{0.85\textwidth}}{\textsf{observador #1}\ensuremath{(#2):#3}}\\%
}
\newcommand{\observadorconreq}[3]{
    & \multicolumn{2}{p{0.85\textwidth}}{\textsf{observador #1}\ensuremath{(#2):#3 \{}}\\
}
\newcommand{\observadorconreqfin}{
    & \multicolumn{2}{p{0.85\textwidth}}{\ensuremath{\}}}\\
}
\newcommand{\obsrequiere}[2][]{& & \textsf{requiere #1: }\ensuremath{#2};\\}

\newcommand{\explicacion}[1]{&& #1 \\}
\newcommand{\invariante}[1]{%
    & \multicolumn{2}{p{0.85\textwidth}}{\textsf{invariante }\ensuremath{#1}}\\%
}
\newcommand{\auxinvariante}[2]{
    & \multicolumn{2}{p{0.85\textwidth}}{\textsf{aux }\ensuremath{#1 = #2}};\\
}

\newcommand{\tiponom}[1]{\ensuremath{\mathsf{#1}}\xspace}
\newcommand{\obsnom}[1]{\ensuremath{\mathsf{#1}}}

% -----------------
% Ecuaciones de terminacion en funcional
% -----------------

\newenvironment{ecuaciones}{%
    $$
    \begin{array}{l @{\ /\ (} l @{,\ } l @{)\ =\ } l}
}{%
    \end{array}
    $$
}

\newcommand{\ecuacion}[4]{#1 & #2 & #3 & #4\\}

% Listas por comprension. El primer parametro es la expresion y el
% segundo tiene los selectores y las condiciones.
\newcommand{\comp}[2]{[\,#1\,|\,#2\,]}


% Encabezado
\lhead{Algoritmos y Estructuras de Datos I}
\rhead{Grupo 07}
% Pie de pagina
\renewcommand{\footrulewidth}{0.4pt}
\lfoot{Facultad de Ciencias Exactas y Naturales}
\rfoot{Universidad de Buenos Aires}

\begin{document}

% Datos de caratula
\materia{Algoritmos y Estructuras de Datos I}
\titulo{Trabajo Pr\'actico N\'umero 2}
%\subtitulo{}
\grupo{Grupo: 07}

\integrante{Demartino, Francisco}{348/14}{demartino.francisco@gmail.com}
\integrante{Frachtenberg Goldsmit, Kevin}{247/14}{kevinfra94@gmail.com}
\integrante{Gomez, Horacio}{756/13}{horaciogomez.1993@gmail.com}
\integrante{Pondal, Iván}{078/14}{ivan.pondal@gmail.com}

\maketitle

\newpage

% Para crear un indice
%\tableofcontents

% Forzar salto de pagina
\clearpage

% Pueden modularizar el documento incluyendo otros .tex
% \include{observaciones}

% \section{Observaciones}
%
% 	\begin{enumerate}
% 		\item un item
% 		\item otro item
% 	\end{enumerate}

% Otro salto de pagina
% \newpage

\section{Especificación}

\begin{ejercicio}

	\begin{problema}{posicionesMasOscuras}{i:Imagen}{res:[(\ent, \ent)]}
		\asegura{mismos(res, [(x,y) \ | \ y \leftarrow [0..alto(i)),x \leftarrow [0..ancho(i)),
		\newline sumaCanalesPixel(color(i, x, y)) == menorSumaCanales(i)]) }
	\end{problema}

\end{ejercicio}

\begin{ejercicio}
	\begin{problema}{top10}{g: Galeria}{result : [Imagen]}
		\asegura[topDiezOMenos]{|result|==min(|imagenes(g)|,10)}
		\asegura[estanEnGaleria]{(\forall im \leftarrow result)im \in imagenes(g)}
		\asegura[sinRepetidos]{(\forall i, j \leftarrow [0..|result|), i \neq j) result[i] \neq result[j]}
		\asegura[ordenPorVotos]{
			(\forall i \leftarrow [0..|result|)) \newline votos(g,result[i])==listaVotosOrdenados(g, imagenes(g))[i]
		}
	\end{problema}
\end{ejercicio}

\begin{ejercicio}
	\begin{problema}{laMasChiquitaConPuntoBlanco}{g:Galeria}{result:Imagen}
		\requiere { |listaImagenesConPixelBlanco(imagenes(g))|>0 }
		\asegura [tienePuntoBlanco]{result \in listaImagenesConPixelBlanco(imagenes(g))}
		\asegura [esChiquita]{area(result) == minimo( \newline [area(i) | i \leftarrow listaImagenesConPixelBlanco(imagenes(g))]) }
	\end{problema}
\end{ejercicio}

\begin{ejercicio}

	\begin{problema*}{agregarImagen}{g:Galeria, i: Imagen}
		\modifica{g}
		\asegura[lasDeAntesEstanConSusVotos] {
			(\forall j \leftarrow imagenes(pre(g))) \newline
			j \in imagenes(g) \land votos(g, j) == votos(pre(g), j) }
		\asegura[lasQueEstanSalvoLaQueAgregoVienenDeAntes] {
			(\forall j \leftarrow imagenes(g), j \neq i) \newline
			j \in imagenes(pre(g))
		}
		\asegura [siAgregoNuevaEntraConCeroVotos]{ i \notin imagenes(pre(g)) \rightarrow \newline ( i \in imagenes(g) \land votos(g, i) == 0) }
	\end{problema*}

\end{ejercicio}

\begin{ejercicio}
      \begin{problema*}{votar}{g: Galeria, i: Imagen}
		\requiere[noSeVotaCualquierCosa]{i \in imagenes(pre(g)}
		\modifica{g}
		\asegura[noCambianLasImagenes]{ mismos(imagenes(g), imagenes(pre(g))) }
		\asegura[noSeTocanLosVotosDeLosOtros]{ (\forall m \leftarrow imagenes(g), m \neq i)\ votos(g, m) == votos(pre(g), m)}
		\asegura[elQueSeVotaSumaUno]{ votos(g,i) == votos(pre(g), i) + 1 }
      \end{problema*}

\end{ejercicio}

\begin{ejercicio}

	\begin{problema*}{eliminarMasVotada}{g: Galeria}
		\requiere[noVacia]{|imagenes(pre(g))| > 0}
		\modifica{g}
		\asegura[seVaUnaSola]{|imagenes(g)| == |imagenes(pre(g))| - 1 }
		\asegura[laQueSeFueEraGrosa]{ (\forall i \leftarrow imagenes(pre(g)),\ i \notin imagenes(g)) \newline
			votos(pre(g), i) == maximo(todosLosVotos(pre(g))) }
		\asegura[lasQueEstanVienenDeAntesConSusVotos] {
			(\forall i \leftarrow imagenes(g)) \newline
			i \in imagenes(pre(g)) \land
			votos(g, i) == votos(pre(g), i)
		}

	\end{problema*}


\end{ejercicio}

% \subsection{Ejercicio X}
\newpage
\subsection{Auxiliares}

\begin{itemize}
	\item \auxil{cuenta(x:T, a:\TLista{T}) : \ent}{
			\longitud {
			  \TLista{1 \ | \ y \leftarrow a, y==x}
			}
	}

	\item \auxil{mismos(a, b: \TLista{T}) : Bool}{
		|a| == |b| \land (\forall c \leftarrow a) cuenta(c,a) == cuenta(c,b)
	}

	\item \auxil{minimo(l : [\ent]) : \ent}{
	   [x \ | \ x \leftarrow l ,(\forall y \leftarrow l) x \leq y)][0]
	}

	\item \auxil{maximo(l : [\ent]) : \ent}{
		[x \ | \ x \leftarrow l ,(\forall y \leftarrow l) x \geq y)][0]
	}

	\item \auxil{sumaCanalesPixel(p : Pixel) : \ent}{
	   red(p) + green(p) + blue(p)
	}

	\item \auxil{listaSumaCanalesPixeles(i: Imagen) : \TLista{\ent}}{ \newline
	  [ sumaCanalesPixel(color(i, x, y))\ | \ y \leftarrow [0..alto(i)), x \leftarrow [0..ancho(i))]
	}

	\item \auxil{area(i: Imagen) : \ent}{
		ancho(i) * alto(i)
	}

	\item \auxil{menorSumaCanales(i: Imagen) : \ent}{
	  minimo(listaSumaCanalesPixeles(i))
	}

	\item \auxil{todosLosVotos(g: Galeria) : [\ent]}{
	  [votos(g, i) | i \leftarrow imagenes(g)]
	}

	\item \auxil{esPixelBlanco(px: Pixel) : Bool}{
	  red(px) == green(px) == blue(px) == 255
	}

	\item \auxil{tienePixelBlanco(i: Imagen) : Bool}{ \newline
	  alguno([esPixelBlanco(color(i,x,y)) | y \leftarrow [0..alto(i)),x \leftarrow [0..ancho(i))])
	}

	\item \auxil{listaImagenesConPixelBlanco(imgs: [Imagen]) : [Imagen]}{
	  [im \ | \ im \leftarrow imgs, tienePixelBlanco(im)]
	}

	\item \auxil{cuentaMasVotos(g: Galeria, imgs: [Imagen], img: Imagen) : \ent}{ \newline
	  |[1 \ | \ im \leftarrow imgs, votos(g,im)>votos(g,img)]|
	}

	\item \auxil{listaVotosOrdenados(g: Galeria, imgs: [Imagen]) : [\ent]}{ \newline
	  [votos(g,im) \ | \ i \leftarrow [0..|imgs|),im \leftarrow imgs, cuentaMasVotos(g,imgs,im)==i]
	}

\end{itemize}

\section{Demostraciones}

\subsection{Predicado de abstracción e invariante de representación}

\begin{lstlisting}
class GaleriaImagenes {
 public:
  void dividirYAgregar(const Imagen &imagen, int n, int m);
  Imagen laMasChiquitaConPuntoBlanco() const;
  void agregarImagen(const Imagen &imagen);
  void votar(const Imagen &imagen);
  void eliminarMasVotada();
  vector <Imagen> top10() const;
  void guardar(std::ostream& os) const;
  void cargar (std::istream& is);

 private:
  void acomodar();
  bool existeImagen(const Imagen &imagen);
  std::vector<Imagen> imagenes;
  std::vector<int> votos;

  // InvRep(imp : GaleriaImagenes):
  // $|imp.imagenes| == |imp.votos| \ \land$
  // $(\forall v \leftarrow imp.votos) \ v \geq 0 \ \land$
  // $(\forall i,j \leftarrow [0..|imp.imagenes|),i < j) \ imp.imagenes[i] \neq imp.imagenes[j] \ \land \ imp.votos[i] \leq imp.votos[j] $

  // abs(imp : GaleriaImagenes, esp : Galeria):
  // $mismos(imp.imagenes, imagenes(esp)) | \ \land$
  // $(\forall i \leftarrow [0..|imp.votos|)) \ imp.votos[i] == votos(esp, imp.imagenes[i])$
};
\end{lstlisting}

\end{document}
